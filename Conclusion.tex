\chapter{Zusammenfassung}
Mit der Hamaube ist ein voll funktionsfähiger Prototyp eines skalierbar
und erweiterbaren Twitter-Klons in dem NoSQL-Projekt entstanden. Er kann
unter \url{http://nosqlprojekt.de} im VPN des Fachbereichs Informatik genutzt
werden. Dort findet ein Frontend zur Hamaube.

Die Hamaube wurde in einem mächtigen Technologiestack aus Kafka,
Cassandra, Elasticsearch und Spark mithilfe von Mikroservices
implementiert.

Die Architektur wurde so gewählt, dass zum einen Skalierbarkeit durch
Scale-Out möglich ist als auch Erweiterbarkeit und Wandelbarkeit der
Anwendung mit geringen Mehraufwand erreicht werden kann.

Die Hamaube hat während des Entwicklungsprozess ihre Erweiterbarkeit
unter Beweis gestellt. Funktionalität konnte kontinuierlich mittels
Kafka leicht hinzugefügt und geupdatet werden. Ebenso konnte die
Kommunikation zwischen den Mikroservices leicht durch Hinzufügen von
Listenern überprüft werden. Dabei spielte es keine Rolle, ob die
Dienste auf den virtuellen Maschinen liefen oder von den
Entwicklungsnotebooks gestartet wurden. Durch Kafka war dies für die
anderen Komponenten transparent.

Der ultimative Scale-out-Test fehlt noch. Die Komponenten sind zwar
skalierbar und auch die Mikroservices skalierbar ausgelegt und mit den
Komponenten verbunden. Da die virtuellen Maschinen aber nur auf einem
Server liegen, konnte der ultimative Lasttest nicht erfolgen. Getestet
wurde das System mit 80 Tweets pro Sekunde über einige Tage. Dies
konnte es gut verarbeiten. Der Test wurde wegen zuneige gehender
Festplattenkapazität dann beendet. Große Datenmenge und eine hohe
Geschwindigkeit machten dem System in diesem Test keine Probleme.

Mit der Hamaube haben wir ein sehr mächtiges Architekturparadigmum
kennengelernt. Wir haben gelernt, wie man komplexe Anwendungen sowohl
in ihre einzelnen Funktionalitäten zerlegt, als auch wie diese in
entsprechenden skalierbaren Tools implementiert werden. Diese einzelnen
Komponenten haben wir dann zu einem funktionierenden Gesamtsystem
zusammengebaut. Wir haben gelernt, wie sich Anwendungen entwickeln
lassen, die sowohl im Sinne von Big Data als auch in flexibler
Langlebigkeit bestehen können.

Natürlich gab es auch einige Hindernisse. Zum einen sind die
eingesetzten Technologien vergleichsweise neu und sind dementsprechend
noch nicht über viele Jahrzehnte gereift, so wie es SQL-Datenbanken
sind. Von Version zu Version kann es dadurch zu größeren Änderungen in
der API kommen. Zum Teil sind diese Tools auch noch unvollständig
dokumentiert. Das Debugging einer solchen verteilten Anwendung ist 
deutlich komplexer als bei einer zentralisierten Anwendung. Es fehlen
sowohl Tools als auch lässt sich das Gesamtsystem nicht einfach
synchron anhalten.

Die Verwendung von autonomen Datenquellen, in unserem Fall den Tweets
von Twitter über die Twitter-API bringt weitere Herausforderungen mit
sich. Das System wird so nie fertig, da sich die Datenquelle jederzeit
ändern kann. Dadurch entsteht eine Notwendigkeit für Flexibilität.
Damit kommt die verwendete Architektur gut klar. Allerdings wird auch
deutlich, dass auch bei Mikroservices eine gewisse Kopplung
(Abhängigkeit) der Komponenten untereinander herrscht.

Als zukünftige Aufgabe lässt sich das automatische Deployment der
Services benennen. Hier existieren bereits heute einige
vielversprechende Ansätze und Tools. Dann könnte man mit viel mehr
Hardware auch den ultimativen Scale-Out-Test durchführen.

Vielen Dank dir Steffen für das interessante Projekt. Wir haben viel Spaß gehabt.