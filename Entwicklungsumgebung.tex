Entwicklungsumgebung NoSQL Master Proejkt 2017/18

Unsere Entwicklungsumgebung ist in einer Private Cloud aufgebaut, die mit Hilfe von OpneStack Infrastruktur-as-a-Server realisiert und auf 6 virtuelle Maschinen aufgeteilt ist. 
Die Micro Services, die ihn ihrer Gesamtheit den Twitter Clone darstellen befinden sich auf dem [134.100.11.144] Knoten. Diese greifen auf die darunterliegende Implementation einer bestimmten Datenbank, Such- und Analytics-Engine: Die Apache Cassandra Datenbank [134.100.11.145], die Such-Engine Elasticsearch [134.100.11.157] sowie die Analytics-Engine Apache Spark [134.100.11.159]. Diese wurden auf eigenständige Knoten aufgeteilt um die Gesamtlast besser zu verteilen, unterbrechungsfreie Entwicklung für jeden Service zu gewährleisten und entkoppelt voneinander zu verwalten.
Jeder Micro Service kommuniziert mit dem Apache Kafka Knoten [134.100.11.143] und ist an mehrere Kafka Topic's angebunden. Damit abonniert jeder Micro Service mindestens ein Topic für die Aufgabenstellung und veröffentlicht die berechnete Ausgabe an einem Ausgabe-Topic.
Die Quelle der Aufgabe sowie die Weiterverarbeitung des Ergebnisses ist von jedem Micro Service gelöst und wird nicht weiterverfolgt. 
Das Frontend wurde mit dem JavaScript Framework Angular und dem WebServer NodeJS auf dem Knoten [134.100.11.58] entwickelt. Für die Demonstration der Skalierbarkeit der Applikation wurden zwei zusätzliche NodeJS Instanzen aufgesetzt, um das richtige Routing der Anfragen über die verschiedenen WebServer zu Visualisieren. 

