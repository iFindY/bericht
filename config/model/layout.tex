\documentclass[11pt,a4paper]{report}
% sostituire con report/book per il formato elettronico

\usepackage[nottoc,notlof,notbib,notindex]{tocbibind} % include l'elenco delle figure e la bibliografia nell'indice.

\usepackage{appendix}
\usepackage{array}
\usepackage{multirow}
\usepackage[ngerman]{babel}
\usepackage{ucs} %unicode sistema gli accenti
\usepackage[utf8x]{inputenc} %unicode sistema gli accenti utf8x
\usepackage[bottom]{footmisc} % posiziona le note in footer sempre in basso
\usepackage{fancyhdr}
\usepackage{graphicx}
\usepackage{color}
%\usepackage{subfigure} % per figure affiancate
\usepackage{subcaption}
\usepackage{supertabular} % used to break tables
\usepackage{float} % per far bene le figures
%\usepackage{indentfirst}
\usepackage[Lenny]{fncychap} % per cambiare i capitoli
\usepackage{longtable} % piazza le note nelle tabelle fuori dalla tabella e permette tabelle che spannano su più pagine
\usepackage{lastpage} % total page count
\usepackage[hyphens]{url}
%\usepackage{breakurl}
\usepackage{tikz}
\usepackage{verbatim}
\usepackage{amsmath}
\usepackage{amssymb}
\usepackage{comment}
\usepackage[colorlinks=true, pdfstartview=FitV, linkcolor=blue, citecolor=blue, urlcolor=blue,breaklinks=true]{hyperref}
\usepackage{listings}
\usepackage{xcolor}
\usepackage{booktabs}


% colori e variabili
% color definitions
\definecolor{red}{rgb}{0.9,0.1,0.1}
\definecolor{blue}{rgb}{0.07,0.55,0.73}
\definecolor{purple}{rgb}{0.4,0.3,0.4}
\definecolor{deep}{rgb}{0.1,0.07,0.3}
\definecolor{white}{rgb}{0.9,0.8,0.86}
% per modificare il colore dei link andare su layout.tex
%layout legacy commands
\renewcommand{\sectionmark}[1]{\markright{\thesection.\ #1}}
\renewcommand{\chaptermark}[1]{\markboth{\thechapter.\ #1}{}}

%user defined commands

% defnisce una pagina bianca con stile plain (migliore delle pagine bianche inserite in automatico dal model book)
\newcommand{\blankpage}{
	\newpage
	% toglie la barra alta dalla pagina vuota
	\thispagestyle{plain}
	% forza una pagina vuota
	\mbox{}
	\newpage
}

%comando per inserire la premessa nel documento (fuori indice)
\newcommand{\premise}[1][]{
	\renewcommand{\theenumi}{#1\roman{enumi}}
	\renewcommand{\labelenumi}{(\theenumi)}
	\thispagestyle{plain}
}


%comandi creati per le convenzioni del documento:
\newcommand{\istage}{\textit{stage}}
\newcommand{\iStage}{\textit{Stage}}
\newcommand{\idfda}{\textit{D.F.D. Assessment System}}
\newcommand{\idfd}{\textit{D.F.D. Consulting}}
\newcommand{\iIT}{\textit{IT}}
\newcommand{\iICT}{\textit{ICT}}
\newcommand{\ibusiness}{\textit{business}}





% indice analitico (comandi con la 'i' scrivono e aggiungono, quelli con 'ii' aggiungono solo, quelli senza niente scrivono solo)
\newcommand{\iASI}{ASI\index{ASI}} %scrive C sharp

%tecnologie
\newcommand{\CSharp}{C♯} %scrive C sharp
\newcommand{\iCSharp}{C♯\index{C♯}} % scrive C sharp e aggiunge l'indice %\unichar{9839}
\newcommand{\idotNET}{.NET\index{.NET}} % scrive .NEt e aggiunge l'indice
\newcommand{\iiWPF}{\index{WPF}} % aggiunge solo l'indice a WPF
\newcommand{\iWPF}{WPF\index{WPF}} % scrive WPF e aggiunge l'indice
\newcommand{\iiWF}{\index{WinForms}}
\newcommand{\iAW}{ANTLRWorks\index{ANTLRWorks}}
\newcommand{\iA}{ANTLR\index{ANTLR}}
\newcommand{\iU}{Unicode\index{Unicode}}
\newcommand{\iPSharp}{\texttt{PdfSharp}\index{PdfSharp@\texttt{PdfSharp}}}
\newcommand{\iTSharp}{\texttt{iTextSharp}\index{iTextSharp@\texttt{iTextSharp}}}

%grammatica
\newcommand{\iiCFG}{\index{CFG (context-free grammar)}}

%programmi
\newcommand{\iVS}{Visual Studio\index{Visual Studio}}
\newcommand{\iVSS}{Visual SourceSafe\index{Visual SourceSafe}}
\newcommand{\iSS}{SQL Server\index{SQL Server}}

% ciclo attivo e passivo
\newcommand{\iCA}{ciclo attivo\index{ciclo!attivo}} % scrive ciclo attivo e aggiunge l'indice
\newcommand{\iCP}{ciclo passivo\index{ciclo!passivo}} % scrive ciclo passivo e aggiunge l'indice
\newcommand{\iiCA}{\index{ciclo!attivo}} % aggiunge l'indice
\newcommand{\iiCP}{\index{ciclo!passivo}} % aggiunge l'indice
\newcommand{\iDCA}{ciclo attivo\index{documento!ciclo attivo}} % scrive ciclo attivo agigunge l'indice al ciclo attivo del documento
\newcommand{\iDCP}{ciclo passivo\index{documento!ciclo passivo}} % scrive ciclo passivo aggiunge l'indice al ciclo passivo del documento
\newcommand{\iiDCA}{\index{documento!ciclo attivo}} % aggiunge l'indice al ciclo attivo del documento
\newcommand{\iiDCP}{\index{documento!ciclo passivo}} % aggiunge l'indice al ciclo passivo del documento
\newcommand{\iGCA}{ciclo attivo\index{gestione!ciclo attivo}} % scrive ciclo attivo e aggiunge l'indice a gestione
\newcommand{\iGCP}{ciclo passivo\index{gestione!ciclo passivo}} % scrive ciclo passivo e aggiunge l'indice a gestione
\newcommand{\iiGCP}{\index{gestione!ciclo passivo}} % aggiunge l'indice a gestione

%componenti
\newcommand{\icMM}{\texttt{MapManager}\index{MapManager@\texttt{MapManager}}}
\newcommand{\iicMM}{\index{MapManager@\texttt{MapManager}}}
\newcommand{\icMF}{\texttt{MapFinder}\index{MapFinder@\texttt{MapFinder}}}
\newcommand{\iicMF}{\index{MapFinder@\texttt{MapFinder}}}
\newcommand{\icPA}{\texttt{PdfAnalyzer}\index{PdfAnalyzer (componente)@\texttt{PdfAnalyzer} (componente)}}

%package e classi
\newcommand{\iPFE}{\texttt{Plain.File.Extraction}\index{Plain.File.Extraction@\texttt{Plain.File.Extraction}}}
\newcommand{\iiPFE}{\index{Plain.File.Extraction@\texttt{Plain.File.Extraction}}}
\newcommand{\iPA}{\texttt{PdfAnalyzer}\index{Plain.File.Extraction@\texttt{Plain.File.Extraction}!PdfAnalyzer (classe)@\texttt{PdfAnalyzer} (classe)}}
\newcommand{\iPP}{\texttt{PdfPage}\index{Plain.File.Extraction@\texttt{Plain.File.Extraction}!PdfPage@\texttt{PdfPage}}}
\newcommand{\iPPar}{\texttt{PdfTextStreamParser}\index{Plain.File.Extraction@\texttt{Plain.File.Extraction}!PdfTextStreamParser@\texttt{PdfTextStreamParser}}}
\newcommand{\iPLex}{\texttt{PdfTextStreamLexer}\index{Plain.File.Extraction@\texttt{Plain.File.Extraction}!PdfTextStreamLexer@\texttt{PdfTextStreamLexer}}}
\newcommand{\iPF}{\texttt{PdfFont}\index{Plain.File.Extraction@\texttt{Plain.File.Extraction}!PdfFont@\texttt{PdfFont}}}
\newcommand{\iPT}{\texttt{PdfText}\index{Plain.File.Extraction@\texttt{Plain.File.Extraction}!PdfText@\texttt{PdfText}}}
\newcommand{\iPC}{\texttt{PdfChar}\index{Plain.File.Extraction@\texttt{Plain.File.Extraction}!PdfChar@\texttt{PdfChar}}}

%parti di un PDF
\newcommand{\iH}{\texttt{Header}\index{PDF!Header@\texttt{Header}}}
\newcommand{\iFT}{\texttt{File Trailer}\index{PDF!File Trailer@\texttt{File Trailer}}}
\newcommand{\iCRTable}{\texttt{Cross Reference Table}\index{PDF!Cross Reference Table@\texttt{Cross Reference Table}}}
\newcommand{\iB}{\texttt{Body}\index{PDF!Body@\texttt{Body}}}

% stream
\newcommand{\iS}{stream\index{stream}}
\newcommand{\iiS}{\index{stream}}
\newcommand{\iCS}{stream\index{content stream}}
\newcommand{\iiCS}{\index{content stream}}

%fasi dello stage
\newcommand{\iifS}{\index{studio del dominio!fase di}}
\newcommand{\iifA}{\index{analisi!fase di}}
\newcommand{\iifP}{\index{progettazione!fase di}}
\newcommand{\iifC}{\index{codifica!fase di}}
\newcommand{\iifV}{\index{verifica e validazione!fase di}}
\newcommand{\iifD}{\index{documentazione!fase di}}

%attività dello stage
\newcommand{\iiaS}{\index{studio del dominio!attività di}}
\newcommand{\iiaA}{\index{analisi!attività di}}
\newcommand{\iiaP}{\index{progettazione!attività di}}
\newcommand{\iiaC}{\index{codifica!attività di}}
\newcommand{\iiaV}{\index{verifica e validazione!attività di}}
\newcommand{\iiaD}{\index{documentazione!attività di}}

%matrici
\newcommand{\Tm}{$T_{m}$}
\newcommand{\Tlm}{$T_{lm}$}

%peratori
\newcommand{\Tc}{\texttt{Tc}\index{operatore!di stato}}
\newcommand{\Tw}{\texttt{Tw}\index{operatore!di stato}}
\newcommand{\Tz}{\texttt{Tz}\index{operatore!di stato}}
\newcommand{\TL}{\texttt{TL}\index{operatore!di stato}}
\newcommand{\Tf}{\texttt{Tf}\index{operatore!di stato}}
\newcommand{\Tr}{\texttt{Tr}\index{operatore!di stato}}
\newcommand{\Ts}{\texttt{Ts}\index{operatore!di stato}}

\newcommand{\Td}{\texttt{Td}\index{operatore!di posizionamento}}
\newcommand{\Tmm}{\texttt{Tm}\index{operatore!di posizionamento}}
\newcommand{\Tstar}{\texttt{T*}\index{operatore!di posizionamento}}

\newcommand{\Tj}{\texttt{Tj}\index{operatore!di stampa}}
\newcommand{\Tquote}{\texttt{'}\index{operatore!di stampa}}
\newcommand{\Tdblquote}{\texttt{\textquotedbl}\index{operatore!di stampa}}
\newcommand{\TJ}{\texttt{TJ}\index{operatore!di stampa}}











\graphicspath{{./pics/}} % cartella di salvataggio immagini

\pagestyle{fancy}

\lhead{\nouppercase{\rightmark}}
\rhead{\nouppercase{\leftmark}}

\fancypagestyle{plain}{
	\lhead{}
	\chead{}
	\rhead{}
	\lfoot{}
	\cfoot{\thepage}
	\rfoot{}
	\renewcommand{\headrulewidth}{0.0pt}% this command should not be add to variables.tex
	\renewcommand{\footrulewidth}{0.1pt}% this command should not be add to variables.tex
}

\fancypagestyle{blank}{
	\lhead{}
	\chead{}
	\rhead{\nouppercase{\rightmark}}
	\lfoot{}
	\cfoot{\thepage}
	\rfoot{}
	\renewcommand{\headrulewidth}{0.1pt}% this command should not be add to variables.tex
	\renewcommand{\footrulewidth}{0.1pt}% this command should not be add to variables.tex
}



% parte per il report
 %	\lhead{}
% 	\chead{}
% 	\rhead{\nouppercase{\rightmark}}
% 	\lfoot{}
% 	\cfoot{\thepage}
% 	\rfoot{}
% 	\renewcommand{\headrulewidth}{0.1pt}% this command should not be add to variables.tex
% 	\renewcommand{\footrulewidth}{0.1pt}% this command should not be add to variables.tex
 
 %\hypersetup{
 %	colorlinks=true,% false: boxed links; true: colored links
% 	linkcolor=blue,% color of internal links
% 	urlcolor=blue,% color of external links
% 	anchorcolor = blue,
% 	citecolor = blue
 %}
%parte per il report



% parte per il book
\fancyhead{}
%\fancyhead[EL]{\nouppercase{\leftmark}}
\fancyhead[OR]{\nouppercase{\rightmark}}
%\fancyfoot[EC,OC]{\thepage}

	\renewcommand{\headrulewidth}{0.1pt}% this command should not be add to variables.tex
	\renewcommand{\footrulewidth}{0.1pt}% this command should not be add to variables.tex
\hypersetup{
	colorlinks=true,% false: boxed links; true: colored links
	linkcolor=black,% color of internal links
	urlcolor=black,% color of external links
	anchorcolor = black,
	citecolor = black
}
%parte per il book
\newcommand{\changelocaltocdepth}[1]{%
	\addtocontents{toc}{\protect\setcounter{tocdepth}{#1}}%
	\setcounter{tocdepth}{#1}%
}

\setlength{\parindent}{0pt}

% indice analitico
\usepackage{makeidx}
\makeindex 

\let\textquotedbl=" % use to print also " in the code

\bibliographystyle{plain}%bibliografia stile inglese

\pagenumbering{Roman}
% fine layout

\newcommand{\figuresource}[1]{\small \par Quelle: #1}

\colorlet{punct}{red!60!black}
\definecolor{background}{HTML}{EEEEEE}
\definecolor{delim}{RGB}{20,105,176}
\colorlet{numb}{magenta!60!black}

\lstdefinelanguage{json}{
	basicstyle=\normalfont\ttfamily,
	numbers=left,
	numberstyle=\scriptsize,
	stepnumber=1,
	numbersep=8pt,
	showstringspaces=false,
	breaklines=true,
	frame=lines,
	backgroundcolor=\color{background},
	literate=
	{0}{{{\color{numb}0}}}{1}
	{1}{{{\color{numb}1}}}{1}
	{2}{{{\color{numb}2}}}{1}
	{3}{{{\color{numb}3}}}{1}
	{4}{{{\color{numb}4}}}{1}
	{5}{{{\color{numb}5}}}{1}
	{6}{{{\color{numb}6}}}{1}
	{7}{{{\color{numb}7}}}{1}
	{8}{{{\color{numb}8}}}{1}
	{9}{{{\color{numb}9}}}{1}
	{:}{{{\color{punct}{:}}}}{1}
	{,}{{{\color{punct}{,}}}}{1}
	{\{}{{{\color{delim}{\{}}}}{1}
	{\}}{{{\color{delim}{\}}}}}{1}
	{[}{{{\color{delim}{[}}}}{1}
	{]}{{{\color{delim}{]}}}}{1},
}

\lstdefinelanguage{Java}{
	basicstyle=\normalfont\ttfamily,
	numbers=left,
	numberstyle=\scriptsize,
	stepnumber=1,
	numbersep=8pt,
	showstringspaces=false,
	breaklines=true,
	frame=lines,
	backgroundcolor=\color{background},
	literate=
	{0}{{{\color{numb}0}}}{1}
	{1}{{{\color{numb}1}}}{1}
	{2}{{{\color{numb}2}}}{1}
	{3}{{{\color{numb}3}}}{1}
	{4}{{{\color{numb}4}}}{1}
	{5}{{{\color{numb}5}}}{1}
	{6}{{{\color{numb}6}}}{1}
	{7}{{{\color{numb}7}}}{1}
	{8}{{{\color{numb}8}}}{1}
	{9}{{{\color{numb}9}}}{1}
	{:}{{{\color{punct}{:}}}}{1}
	{,}{{{\color{punct}{,}}}}{1}
	{\{}{{{\color{delim}{\{}}}}{1}
	{\}}{{{\color{delim}{\}}}}}{1}
	{[}{{{\color{delim}{[}}}}{1}
	{]}{{{\color{delim}{]}}}}{1},
}

\lstdefinelanguage{SQL}{
	basicstyle=\normalfont\ttfamily,
	numbers=left,
	numberstyle=\scriptsize,
	stepnumber=1,
	numbersep=8pt,
	showstringspaces=false,
	breaklines=true,
	frame=lines,
	backgroundcolor=\color{background},
	literate=
	{0}{{{\color{numb}0}}}{1}
	{1}{{{\color{numb}1}}}{1}
	{2}{{{\color{numb}2}}}{1}
	{3}{{{\color{numb}3}}}{1}
	{4}{{{\color{numb}4}}}{1}
	{5}{{{\color{numb}5}}}{1}
	{6}{{{\color{numb}6}}}{1}
	{7}{{{\color{numb}7}}}{1}
	{8}{{{\color{numb}8}}}{1}
	{9}{{{\color{numb}9}}}{1}
	{:}{{{\color{punct}{:}}}}{1}
	{,}{{{\color{punct}{,}}}}{1}
	{\{}{{{\color{delim}{\{}}}}{1}
	{\}}{{{\color{delim}{\}}}}}{1}
	{[}{{{\color{delim}{[}}}}{1}
	{]}{{{\color{delim}{]}}}}{1},
}

\newcommand{\TODO}[1]{\marginpar{\color{red}\emph{\small{{\bf TODO: } #1}}}}
