\documentclass[11pt,a4paper]{report}
% sostituire con report/book per il formato elettronico

\usepackage[nottoc,notlof,notbib,notindex]{tocbibind} % include l'elenco delle figure e la bibliografia nell'indice.

\usepackage{appendix}
\usepackage{array}
\usepackage{multirow}
\usepackage[ngerman]{babel}
\usepackage{ucs} %unicode sistema gli accenti
\usepackage[utf8x]{inputenc} %unicode sistema gli accenti utf8x
\usepackage[bottom]{footmisc} % posiziona le note in footer sempre in basso
\usepackage{fancyhdr}
\usepackage{graphicx}
\usepackage{color}
%\usepackage{subfigure} % per figure affiancate
\usepackage{subcaption}
\usepackage{supertabular} % used to break tables
\usepackage{float} % per far bene le figures
%\usepackage{indentfirst}
\usepackage[Lenny]{fncychap} % per cambiare i capitoli
\usepackage{longtable} % piazza le note nelle tabelle fuori dalla tabella e permette tabelle che spannano su più pagine
\usepackage{lastpage} % total page count
\usepackage[hyphens]{url}
%\usepackage{breakurl}
\usepackage{tikz}
\usepackage{verbatim}
\usepackage{amsmath}
\usepackage{amssymb}
\usepackage{comment}
\usepackage[colorlinks=true, pdfstartview=FitV, linkcolor=blue, citecolor=blue, urlcolor=blue,breaklinks=true]{hyperref}
\usepackage{listings}
\usepackage{xcolor}
\usepackage{booktabs}


% colori e variabili
\input{./config/model/colors.tex}
\input{./config/model/variables.tex}

\graphicspath{{./pics/}} % cartella di salvataggio immagini

\pagestyle{fancy}

\lhead{\nouppercase{\rightmark}}
\rhead{\nouppercase{\leftmark}}

\fancypagestyle{plain}{
	\lhead{}
	\chead{}
	\rhead{}
	\lfoot{}
	\cfoot{\thepage}
	\rfoot{}
	\renewcommand{\headrulewidth}{0.0pt}% this command should not be add to variables.tex
	\renewcommand{\footrulewidth}{0.1pt}% this command should not be add to variables.tex
}

\fancypagestyle{blank}{
	\lhead{}
	\chead{}
	\rhead{\nouppercase{\rightmark}}
	\lfoot{}
	\cfoot{\thepage}
	\rfoot{}
	\renewcommand{\headrulewidth}{0.1pt}% this command should not be add to variables.tex
	\renewcommand{\footrulewidth}{0.1pt}% this command should not be add to variables.tex
}



% parte per il report
 %	\lhead{}
% 	\chead{}
% 	\rhead{\nouppercase{\rightmark}}
% 	\lfoot{}
% 	\cfoot{\thepage}
% 	\rfoot{}
% 	\renewcommand{\headrulewidth}{0.1pt}% this command should not be add to variables.tex
% 	\renewcommand{\footrulewidth}{0.1pt}% this command should not be add to variables.tex
 
 %\hypersetup{
 %	colorlinks=true,% false: boxed links; true: colored links
% 	linkcolor=blue,% color of internal links
% 	urlcolor=blue,% color of external links
% 	anchorcolor = blue,
% 	citecolor = blue
 %}
%parte per il report



% parte per il book
\fancyhead{}
%\fancyhead[EL]{\nouppercase{\leftmark}}
\fancyhead[OR]{\nouppercase{\rightmark}}
%\fancyfoot[EC,OC]{\thepage}

	\renewcommand{\headrulewidth}{0.1pt}% this command should not be add to variables.tex
	\renewcommand{\footrulewidth}{0.1pt}% this command should not be add to variables.tex
\hypersetup{
	colorlinks=true,% false: boxed links; true: colored links
	linkcolor=black,% color of internal links
	urlcolor=black,% color of external links
	anchorcolor = black,
	citecolor = black
}
%parte per il book
\newcommand{\changelocaltocdepth}[1]{%
	\addtocontents{toc}{\protect\setcounter{tocdepth}{#1}}%
	\setcounter{tocdepth}{#1}%
}

\setlength{\parindent}{0pt}

% indice analitico
\usepackage{makeidx}
\makeindex 

\let\textquotedbl=" % use to print also " in the code

\bibliographystyle{plain}%bibliografia stile inglese

\pagenumbering{Roman}
% fine layout

\newcommand{\figuresource}[1]{\small \par Quelle: #1}

\colorlet{punct}{red!60!black}
\definecolor{background}{HTML}{EEEEEE}
\definecolor{delim}{RGB}{20,105,176}
\colorlet{numb}{magenta!60!black}

\lstdefinelanguage{json}{
	basicstyle=\normalfont\ttfamily,
	numbers=left,
	numberstyle=\scriptsize,
	stepnumber=1,
	numbersep=8pt,
	showstringspaces=false,
	breaklines=true,
	frame=lines,
	backgroundcolor=\color{background},
	literate=
	{0}{{{\color{numb}0}}}{1}
	{1}{{{\color{numb}1}}}{1}
	{2}{{{\color{numb}2}}}{1}
	{3}{{{\color{numb}3}}}{1}
	{4}{{{\color{numb}4}}}{1}
	{5}{{{\color{numb}5}}}{1}
	{6}{{{\color{numb}6}}}{1}
	{7}{{{\color{numb}7}}}{1}
	{8}{{{\color{numb}8}}}{1}
	{9}{{{\color{numb}9}}}{1}
	{:}{{{\color{punct}{:}}}}{1}
	{,}{{{\color{punct}{,}}}}{1}
	{\{}{{{\color{delim}{\{}}}}{1}
	{\}}{{{\color{delim}{\}}}}}{1}
	{[}{{{\color{delim}{[}}}}{1}
	{]}{{{\color{delim}{]}}}}{1},
}

\lstdefinelanguage{Java}{
	basicstyle=\normalfont\ttfamily,
	numbers=left,
	numberstyle=\scriptsize,
	stepnumber=1,
	numbersep=8pt,
	showstringspaces=false,
	breaklines=true,
	frame=lines,
	backgroundcolor=\color{background},
	literate=
	{0}{{{\color{numb}0}}}{1}
	{1}{{{\color{numb}1}}}{1}
	{2}{{{\color{numb}2}}}{1}
	{3}{{{\color{numb}3}}}{1}
	{4}{{{\color{numb}4}}}{1}
	{5}{{{\color{numb}5}}}{1}
	{6}{{{\color{numb}6}}}{1}
	{7}{{{\color{numb}7}}}{1}
	{8}{{{\color{numb}8}}}{1}
	{9}{{{\color{numb}9}}}{1}
	{:}{{{\color{punct}{:}}}}{1}
	{,}{{{\color{punct}{,}}}}{1}
	{\{}{{{\color{delim}{\{}}}}{1}
	{\}}{{{\color{delim}{\}}}}}{1}
	{[}{{{\color{delim}{[}}}}{1}
	{]}{{{\color{delim}{]}}}}{1},
}

\lstdefinelanguage{SQL}{
	basicstyle=\normalfont\ttfamily,
	numbers=left,
	numberstyle=\scriptsize,
	stepnumber=1,
	numbersep=8pt,
	showstringspaces=false,
	breaklines=true,
	frame=lines,
	backgroundcolor=\color{background},
	literate=
	{0}{{{\color{numb}0}}}{1}
	{1}{{{\color{numb}1}}}{1}
	{2}{{{\color{numb}2}}}{1}
	{3}{{{\color{numb}3}}}{1}
	{4}{{{\color{numb}4}}}{1}
	{5}{{{\color{numb}5}}}{1}
	{6}{{{\color{numb}6}}}{1}
	{7}{{{\color{numb}7}}}{1}
	{8}{{{\color{numb}8}}}{1}
	{9}{{{\color{numb}9}}}{1}
	{:}{{{\color{punct}{:}}}}{1}
	{,}{{{\color{punct}{,}}}}{1}
	{\{}{{{\color{delim}{\{}}}}{1}
	{\}}{{{\color{delim}{\}}}}}{1}
	{[}{{{\color{delim}{[}}}}{1}
	{]}{{{\color{delim}{]}}}}{1},
}

\newcommand{\TODO}[1]{\marginpar{\color{red}\emph{\small{{\bf TODO: } #1}}}}
