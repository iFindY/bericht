\chapter{Einleitung}
\pagenumbering{arabic}

\section{Kontext der Petrinetze} \label{sec:KdP} 

Das Konzept der Petrinetze wurde in der Arbeit von \textit{Carl Petri} beschrieben. Dieses besteht aus Stellen, Marken, Kanten und Transitionen, die nebenläufige und kommunizierende Prozesse darstellen können. Der ursprüngliche S/T-Netz Formalismus wurde mit der Zeit durch gefärbten Marken erweitert, mit dem Ziel äquivalente Strukturen zusammenzufassen und die darin befindlichen Marken zu typisieren \cite{kummerReferenznetze}. Da die Struktur des Netzes immer noch stark zusammenhängend ist, bleibt die Organisation des Netzes schwer verständlich für das menschliche Auge.\bigbreak

Demzufolge sollte des Netzes auf logisch zusammenhänge Komponente aufgeteilt werden und trotzdem als ein Ganzes gelten. Diese Anforderung wird von den synchronen Kanälen umgesetzt, indem die Netzkomponenten anstelle der Kanten mit synchronen Kanälen verbunden werden und zwingen die mit einander verbunden Transitionen synchron zu schalten \cite{kummerReferenznetze}. Hiermit ist eine Trennung des Netzes nach ihrer Funktionalität erreicht, die qualitativ anspruchsvolle Modelle komplexer und verteilter Systeme entwerfen lässt.\bigbreak

Obwohl das erweiterte Petrinetz anspruchsvolle Modellierungswerkzeug bietet, bleibt das gesamte Netzwerk statisch. Demzufolge wurde der nächste Evolutionsschritt in der Entwicklung der Petrinetze mit den Referenznetz Formalismus getan. Dieser erlaubt dynamisch und bei Bedarf Netzinstanzen zu erstellen und diese als Marken in einem anderen Netz zu bewegen. Somit kann es mehrerer Instanzen eines Netzes geben, die mit unterschiedlicher Belegung im Petrinetz existieren \cite{kummerReferenznetze}.

\section{Ein Petrinetz Simulator} \label{sec:EPS}

Renew ist ein Petrinetz Simulator, der die oben genannten Petrinetz Formalismen unterstützt. Dieser ist in Java geschrieben und bietet eine Oberfläche zum Zeichnen und einen Simulator zum Ausführen der Netze \cite{userGuide}.\bigbreak

Da die ursprüngliche Umsetzung von Olaf Kummer eine empfindliche, monolithische Architektur besaß und viel Fachwissen voraussetzte, wurde diese zu einem Plugin Verband von Jörn Schumacher zu Gunsten der Robustheit und Erweiterung unstrukturiert \cite{schummacher}. Ab diesen Punkt kann Renew über die Plugin Schnittstellen erweitert werden, ohne die existierende Logik zu beeinflussen.\bigbreak

Mit seiner Umsetzung delegierte Jörn Schumacher die Ausführung von Logik an Plugins und erstellte eine zentrale Instanz, die den Lebenszyklus bekannter Plugins verwaltet und koordiniert. Die zentrale Instanz nennt sich Plugin-Manager und kann das Verhalten von Renew mit Hilfe der Plugins modifizieren. Der Plugin-Manager baut auf zwei primären Namensräume auf. Zum einen braucht dieser zusätzliche Bibliotheken zum verwalten seiner Umgebung und zum anderen braucht er Plugins, die Funktionalität mit sich bringen \cite{douvigneau}.

\section{Konstellation des Simulators} \label{sec:KdS}

Mit der Plugin Architektur hat Renew den Lebenszyklus weit überschritten. Dies kann an den Codestellen abgelesen werden, die zum Teil aus der JDK 1.4 Version stammen. Somit entsprechen die erstmaligen Gestaltungsmöglichkeiten, Architekturentscheidungen und ihre Umsetzung, nicht mehr den aktuellen Stand der Technik. Vor allem durch die Einführung des Modulsystem von Java, mit dem der JDK sowie der darauf aufbauende Code modularisiert wird, ist der Betrieb von Renew in der Zukunft gefährdet. Im Zuge dessen ist das Portieren der Applikation unvermeidlich und trägt ein gewisses Risiko mit sich. Es ist unklar wie sich die benutzerdefinierten Namensräume und die so gut wie unberührten Kern-Plugins auf die neuen Modulstruktur übertragen lassen. Zumal die Suche nach zusätzlichen Plugin Code eine zentrale Funktion in System vertretet.\bigbreak

Indem die Applikation neu strukturiert wird, sind weitere Mängel zu erwarten, denn das Vernachlässigen der Gesamtarchitektur und die Konzentration auf eigne Plugins führt zu Code-Duplizierung und Problemen in der Verständlichkeit der Plugin-Abhängigkeiten.\newline
Beifolgend stellt sich die Frage: Wie portierbar ist Renew und was muss getan werden, damit der Umstieg auf das Modulsystem von Java gelingt.

\section{Ziel} \label{sec:Z}

Das Ziel dieser Arbeit ist die Anpassung von Renew an die neuen Anforderungen der Java Laufzeitumgebung, sodass eine minimal und anschließend eine  erweiterte Version von Renew entstehen, die schlussendlich das \textit{Mulan} Framework unterstützt. Dementsprechend soll Renew einen einfachen Einstig in die existierende Kernlogik geben, sowie die Entwickler-Fähigkeiten in der Forschung verteilter Systeme vorantreiben, indem selbst erstellte Simulation verteilt ausgeführt werden dürfen.

\section{Umsetzung} \label{sec:U}

Als Teil der Umsetzung entstehen zwei Prototypen, die ausgewählte Teile von Renew modular restrukturieren. Demzufolge wird die Codebasis sowie Design Entscheidungen bei Bedarf modernisiert und auf das Modulsystem von Java angepasst. Dabei soll der Schwerpunkt dieser Arbeit beim Erarbeiten einer modularen Renew Umsetzung liegen und demnach Plugins in erweiterbare Java Module transformieren.\newline
Dafür wird die bestehende Projekt Konstruktion reorganisiert, sodass Plugins durch mehrere Module repräsentiert wenden können. Darauf aufbauend müssen Plugins die Moduleigenschaften einhalten und die entsprechenden Regeln und Konfigurationen erfüllen. Und zum Schluss wird die Renew Applikation mit einem modernen \textit{build} Werkzeug organisiert und für die Ausführung verpackt.\newline
Das Ergebnis soll den Modularisierungskonzept in jedem Abschnitt des Lebenszyklus von Renew unterstützen. Daher wird erwartet, dass das modulare Denken in der Entwicklungsumgebung beim Entwickeln, beim Kompilieren und Verpacken sowie beim Ausführen der Applikation integriert wird. \bigbreak

%todo 
Das Ergebnis soll als Grundlage für die nächste Lebenszyklus von Renew dienen und die Entwicklung einfach, flexible und erweiterbar gestalten. Dafür wir die Unterstützung von \textit{Mulan} benötigt, welches die wesentliche Funktion von Renew voraussetzt. 

\section{Aufbau der Arbeit} \label{sec:AdA}
Hier kommt der Aufbau 

