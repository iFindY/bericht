\chapter{Gesamt Evaluation}
Dieses Kapitel evaluiert die gesetzten Anforderungen aus der Anforderungsanalyse und der Anforderungsspezifikation und bewertet die Migration auf das Modulsystem von Java.\bigbreak	

In den Grundalgen Kapitel wurde wichtige Konzepte und Einschaften der Modularisierung erarbeitet, die in der Zukunft helfen sollen Module sauber zu entwerfen, zu erstellen oder zu bewerten. Dazu gehören kritische Modul Charakteristika, wie Modulkopplung, Modulbindung, Seiteneffektfreiheit, Modulgröße und Namensräume. \bigbreak

Mithilfe des Migrationskapitels werden zwei wesentliche Vorgehensweisen dargestellt, mit den eine Migration durchgeführt werden kann. Dieses beschreibt unter anderem wie das Modulsystem von Java sich eine Kontinuierliche Migration auf das Modulsystem vorstellt. \newline
Die Migration des Renew Prototypen bediene sich dieser Idee und modularisiert die Plugins entsprechen der \textit{bottom Up} Ansatzes. Darüber hinaus sind wesentliche Migrationshürden auf das Modulsystem von Java benannt worden, die die essenziellen Probleme zusammenfassen. \bigbreak

 




% Anforderungsanalyse/Spezifikation
%- Anforderung an Modulariserte Systeme 
%- Charakterisierung (Aufbau Schnittstellen Umfang)
%- Aufwand für Modularisierung
%- Konsequenzen (Zyklen , Splitt Packages ,Alte APIEs)
%- Konsistenz 
%- Mehrwert
%
%- Übergangsphase Aufwand 
%
%- Gradle -> Projektstruktur 
%- Zusätzliche Module Unterstützen 
%- gelöst von der Entwiklungsumgebung 
%- kohärente Arbeitsweise
%- ältere Verfahren 




% Evaluaiere die Anfprderungsanalyse Übergeordnete Kritik 
% Hat sich das Gradle wärkzueg behauptet ? verstädnlichkeit, code menge, erweiterbarkeit, konfiguration , mehrwert in der nutzung 
% Modularisierung konsequenzen für Systeme 
% 

% Was wollte ich erreichen 
% Unterziele 
% Bewertung der Unterziehle 




% Was habe ich erreicht 


