% Motivation Modularisierung 
Die Digitalisierung innerhalb der Geselschaft durchdringt jeden Bereich unsereres Lebens. Vom Web, Smartphones, \textit{Augmented Reality} bis zu \textit{Smart Cities} wird die Entwiklung und Beherschung komplexer Softwaresysteme eine tragnede Rolle spielen. Dementsprechend spielt der Wartungsaufwand, die Erweiterbarkeit und die Flexibiliätet des Systems eine zentralle Rolle für die Softwareentwiklung. Aktuell bescheftigen sich die Microservices und die Kontainer Technologine mit der Beherschbarkeit des Systems auf der Architekturebene. Jedoch fehlt diesen die Unterstützung beim Entwiklen feingraularer, rekombenriebarer Komponenten auf der Codeebene. Diese wichtige Aufgabe hat sich das Modulsystem von Java gewidmet und ermöglicht das Esrtellen von Modulen die spezifische Aufgaben kapseln und die Komplexität innerhalb der Applikationen reduzieren. In der Konsequnez könenne einzelen Teile einer Applikation problemlos ausgetauscht, getestet, oder erweitert werden, ohne die gesamte Applikation in betrach zu ziehen.\bigbreak

% zu lösendes Problem
\textsc{Renew} ist ein Petrinetz Simulator, der für die Modellierung von komplexen und verteilten Systemen entwickelt worden ist. Dieser wurde in einer einfachen monolithischen Architektur umgesetzt, die sensibel auf Änderungen ungeübter Entwickler reagierte. Aus diesem Grund wurde eine Plugin-Architektur entwickelt, die \textsc{Renew} auf Plugins zerlegte und mit zusätzlicher Funktionalität anreichern lies, ohne ihre innere Struktur zu verletzen. \bigbreak
Obwohl die Plugin-Architektur sich langfristig bewährte, ist sie nicht für die modernen Herausforderungen gewappnet, denn das Modulsystem von Java führt eine obligatorisches Konzept der Module ein, das den Betrieb von Altsoftware 
in absehbarer Zeit aufhebt. Dementsprechend muss \textsc{Renew} sich dieser Herausforderung stellen und an die neue Umgebung anpassen. \newline

% Lösungsansatz
Im ersten Abschnitt diese Arbeit werden Grundlagen der Java Laufzeitumgebung erarbeitet. Anschließend werden aufbauende Konzepte der Modularisierung von Java diskutiert, die im Folgenden die Migration von bestehenden Applikationen auf das Modulsystem von Java einleiten. Für die Umsetzung der erarbeiteten Konzepte entsteht eine Spezifikation, die von zwei Prototypen umgesetzt wird. Zum Schluss folgt eine Evaluation der gesetzten Ziele.


% Ergebnisse TODO
Im ersten Abschnitt diese Arbeit werden Grundlagen der Java Laufzeitumgebung erarbeitet. Anschließend werden aufbauende Konzepte der Modularisierung von Java diskutiert, die im Folgenden die Migration von bestehenden Applikationen auf das Modulsystem von Java einleiten. Für die Umsetzung der erarbeiteten Konzepte entsteht eine Spezifikation, die von zwei Prototypen umgesetzt wird. Zum Schluss folgt eine Evaluation der gesetzten Ziele.

% Fazit
Da die Anzahl der \textsc{Renew} Plugins, ihre Größe sowie ihre Komplexität ständig wächst, steigt der Wartungsaufwand. Dementsprechend ist die Modularisierung von Java essentiell wichtig für die Beherrschung der zunehmenden Komplexität von \textsc{Renew} und damit auch für die Wartbarkeit der langlebigen Architektur.\newline
Darüber hinaus fördert das Modulsystem von Java das parallele und gemeinschaftliche Arbeiten an modularisierten Projekten, die in dem \textsc{Renew} Kontext eine wichtige Rolle spielen.\newline








% In der Konsequnez könenne einzelen Teile einer Applikation problemlos ausgetauscht, getestet, oder erweitert werden, ohne die gesamte Applikation in betrach zu ziehen.

Ziel dieser Arbeit ist die Migration von \textsc{Renew} auf das Modulsystem von Java sowie das Ersetzen des existierendes \textit{Build} Werkzeug durch Gradle, welches eine intelligente Projektverwaltung sowie eine überlege Integration in moderne Technologien verspricht. \newline
Darüber hinaus soll die Modularisierung und die Migration von \textsc{Renew} die Entwicklung einer möglichen nachfolgenden Mikroservice Architektur begünstigen.\bigbreak


Im ersten Abschnitt diese Arbeit werden Grundlagen der Java Laufzeitumgebung erarbeitet. Anschließend werden aufbauende Konzepte der Modularisierung von Java diskutiert, die im Folgenden die Migration von bestehenden Applikationen auf das Modulsystem von Java einleiten. Für die Umsetzung der erarbeiteten Konzepte entsteht eine Spezifikation, die von zwei Prototypen umgesetzt wird. Zum Schluss folgt eine Evaluation der gesetzten Ziele.


Aus Evaluation 
%  Das neu eingeführte Modulsystem von Java widmet sich dieser Aufgabe und verringert die Komplexität beim Entwickeln von Softwaresystemen, indem das Gesatmsystem von Grundauf aus Modulen besteht und einen einfachen Weg für das Erstellen, Austauschen und Warten von Codebausteinen des Systems ermöglicht. 



% Zur diesen Zeitpunkt domenieren die Kontainer und Microservices Technologine den Markt, da sie eine felxible Art und Weise für die Entwklung der komplexen Systeme anbieten. Diese verfolgen die Idee der Aufteiung eienr mnolitischen Architektur in aufgabenbezogene Komponenten, um die Wartbarkeit des gesamtsystems zu erhöhen. Jedoch kann dieser Ansatz, ohne verbindliche Richtlininien sowie der Unterstützung der Laufzeit Plattform, eine schwierige Aufgabe darstellen.


